\documentclass[
  a4paper,
  12pt,
  UTF8,
  AutoFakeBold=2.5,
  fontset=custom,
]{ctexart}  % 见文件 ctex-fontset-custom.def
\usepackage[slantfont, boldfont]{xeCJK}
\usepackage[T1]{fontenc}
\usepackage[english]{babel}
\usepackage[dvipsnames]{xcolor}
\usepackage{graphicx}
\usepackage{amsmath}
\usepackage{amssymb}
\usepackage{wasysym}
\usepackage{pifont}
\usepackage[unicode, hidelinks]{hyperref}
\usepackage{geometry}
\usepackage{relsize}
\usepackage{csquotes}
\usepackage{currfile}
\usepackage{tikz}
\usetikzlibrary{shapes, arrows.meta, positioning, fit}
\usepackage{cleveref}
% customize reference names
\crefname{figure}{图}{图}
\crefname{table}{表}{表}
\crefname{equation}{式}{式}
% \crefname{section}{节}{节}
% \crefname{subsection}{小节}{小节}

% \usepackage{gbt7714}
% \usepackage{natbib}
% \setlength{\bibsep}{0.0pt}
\usepackage[
  backend=biber,
  style=gb7714-2015,
  % style=gb7714-2015ay,
  gbnamefmt=lowercase,  % 作者姓名大小写,"uppercase" 表示全大写,"lowercase" 表示由输入信息确定不做改变,以及其他一些选项,未指定时默认为 "uppercase"
  gbalign=left,  % 参考文献条目标签的对齐方式,未指定时默认为 right
  gbstrict=true,  % 避免输出 bib 文件中多余的域信息,比如 language 等,未指定时默认为 true
  gbtype=true,  % 是否输出题名后面的文献类型和载体标识符,比如 [J]. [M]. [D]. 等,未指定时默认为 true
  gbcitelabel=bracketqj,  % 正文文献标注中的全半角标点和括号,bracket表示采用方括号,qj表示采用全角标点,未指定时默认为 bracket
  gbbiblabel=bracketqj,  % 参考文献表中的标注标点,bracket表示采用方括号,qj表示采用全角标点,未指定时默认为 bracket
  maxbibnames=5,
]{biblatex}
\addbibresource[location=local]{references.bib}
\AtBeginBibliography{\smaller[1]}
\defbibheading{bibliography}[\bibname]{%
  \subsubsection*{\larger[1] \kaishu #1}%
  \markboth{#1}{#1}}
\setlength\bibitemsep{\itemsep}

% % 【著者-出版年制】文献表缩进控制
% \setlength{\bibitemindent}{0em} % bibitemindent表示一条文献中第一行相对后面各行的缩进
% \setlength{\bibhang}{0pt} % 著者-出版年制中 bibhang 表示的各行起始位置到页边的距离
% %
% % 【顺序编码制】文献表缩进控制
% % 调整顺序标签与文献内容的间距
% \setlength{\biblabelsep}{2mm}
% \setlength{\bibitemindent}{0pt}
% \setlength{\biblabelextend}{0pt}

\usepackage{enumitem}
\usepackage{ulem}
\usepackage{lmodern}

% \newCJKfontfamily\simfang{simfang.ttf}[Extension=.ttf, Path=fonts/]
% \newCJKfontfamily\simhei{simhei.ttf}[Extension=.ttf, Path=fonts/]
% \newCJKfontfamily\simkai{simkai.ttf}[Extension=.ttf, Path=fonts/]
% \newCJKfontfamily\simsun{simsun.ttc}[Extension=.ttc, Path=fonts/]
% \setCJKmainfont[Path=fonts/, BoldFont=simhei.ttf, ItalicFont=simkai.ttf]{simsun.ttc}
% \setCJKsansfont[Path=fonts/, BoldFont=simhei.ttf]{simkai.ttf}
% \setCJKmonofont[Path=fonts/, BoldFont=simhei.ttf]{simfang.ttf}
% \renewcommand*{\songti}{\CJKfamily{simsun}}
% \renewcommand*{\heiti}{\CJKfamily{simhei}}
% \renewcommand*{\kaishu}{\CJKfamily{simkai}}
% \renewcommand*{\fangsong}{\CJKfamily{simfang}}

%\geometry{left=3.23cm,right=3.23cm,top=2.54cm,bottom=2.54cm}
%latex的页边距比word的视觉效果要大一些,稍微调整一下
%\geometry{left=2.95cm,right=2.95cm,top=2.54cm,bottom=2.54cm}%2020
%\geometry{left=2.95cm,right=2.95cm,top=2.54cm,bottom=2.54cm}
\geometry{left=3.00cm,right=3.07cm,top=2.67cm,bottom=3.27cm}
\pagestyle{empty}
\setcounter{secnumdepth}{-2} %不让那些section和subsection自带标号,标号格式自己掌握
\definecolor{MsBlue}{RGB}{0,112,192} %Ms Word 的蓝色和latex xcolor包预定义的蓝色不一样。通过屏幕取色得到。
% Renaming floats with babel
\addto\captionsenglish{
    \renewcommand{\contentsname}{目录}
    \renewcommand{\listfigurename}{插图目录}
    \renewcommand{\listtablename}{表格}
    %\renewcommand{\refname}{\sihao 参考文献}
    \renewcommand{\refname}{\sihao \kaishu \leftline{参考文献}} %这几个字默认字号稍大,改成四号字,楷书,居左(默认居中) 根据喜好自行修改,官方模板未作要求
    \renewcommand{\abstractname}{摘要}
    \renewcommand{\indexname}{索引}
    \renewcommand{\tablename}{表}
    \renewcommand{\figurename}{图}
    } %把Figure改成‘图’,reference改成‘参考文献’。如此处理是为了避免和babel包冲突。
%定义字号
\newcommand{\chuhao}{\fontsize{42pt}{\baselineskip}\selectfont}
\newcommand{\xiaochuhao}{\fontsize{36pt}{\baselineskip}\selectfont}
\newcommand{\yihao}{\fontsize{26pt}{\baselineskip}\selectfont}
\newcommand{\erhao}{\fontsize{22pt}{\baselineskip}\selectfont}
\newcommand{\xiaoerhao}{\fontsize{18pt}{\baselineskip}\selectfont}
\newcommand{\sanhao}{\fontsize{16pt}{\baselineskip}\selectfont}
\newcommand{\sihao}{\fontsize{14pt}{\baselineskip}\selectfont}
\newcommand{\xiaosihao}{\fontsize{12pt}{\baselineskip}\selectfont}
\newcommand{\wuhao}{\fontsize{10.5pt}{\baselineskip}\selectfont}
\newcommand{\xiaowuhao}{\fontsize{9pt}{\baselineskip}\selectfont}
\newcommand{\liuhao}{\fontsize{7.875pt}{\baselineskip}\selectfont}
\newcommand{\qihao}{\fontsize{5.25pt}{\baselineskip}\selectfont}
%字号对照表
%二号 21pt
%四号 14
%小四 12
%五号 10.5
%设置行距 1.5倍
\renewcommand{\baselinestretch}{1.5}
% \setlength{\parskip}{0.5\baselinestretch}
\XeTeXlinebreaklocale "zh"           % 中文断行


\newcommand\wordcount{\input{|"texcount -inc -sum -0 -utf8 -ch -template={SUM} \currfilepath"}}
