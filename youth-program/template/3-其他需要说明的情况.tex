{\color{MsBlue} \subsection{\texorpdfstring{\sihao \kaishu \quad \ (三)其他需要说明的情况 }{(三)其他需要说明的情况 }}}

{\sihao \color{MsBlue} \kaishu 1. 申请人同年申请不同类型的国家自然科学基金项目情况(列明同年申请的其他项目的项目类型、项目名称信息,并说明与本项目之间的区别与联系)。 }

无。

\vskip 5mm


{\sihao \color{MsBlue} \kaishu 2. 具有高级专业技术职务(职称)的申请人是否存在同年申请或者参与申请国家自然科学基金项目的单位不一致的情况;如存在上述情况,列明所涉及人员的姓名,申请或参与申请的其他项目的项目类型、项目名称、单位名称、上述人员在该项目中是申请人还是参与者,并说明单位不一致原因。}

无。

\vskip 5mm


{\sihao \color{MsBlue} \kaishu 3. 具有高级专业技术职务(职称)的申请人是否存在与正在承担的国家自然科学基金项目的单位不一致的情况;如存在上述情况,列明所涉及人员的姓名,正在承担项目的批准号、项目类型、项目名称、单位名称、起止年月,并说明单位不一致原因。}

无。

\vskip 5mm

{\sihao \color{MsBlue} \kaishu 4. 其他。}

无。
